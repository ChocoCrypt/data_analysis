% article example for classicthesis.sty
\documentclass[10pt,a4paper]{article} % KOMA-Script article scrartcl
\usepackage{lipsum}     %lorem ipsum text
\usepackage{titlesec}   %Section settings
\usepackage{titling}    %Title settings
\usepackage[margin=10em]{geometry}  %Adjusting margins
\usepackage{setspace}
\usepackage{listings}
\usepackage{amsmath}    %Display equations options
\usepackage{amssymb}    %More symbols
\usepackage{xcolor}     %Color settings
\usepackage{pagecolor}
\usepackage{mdframed}
\usepackage[spanish]{babel}
\usepackage[utf8]{inputenc}
\usepackage{longtable}
\usepackage{multicol}
\usepackage{graphicx}
\graphicspath{ {./Images/} }
\setlength{\columnsep}{1cm}

% ====| color de la pagina y del fondo |==== %
\pagecolor{black}
\color{white}



\begin{document}
    %========================{TITLE}====================%
    \title{\rmfamily\normalfont\spacedallcaps{ Tarea segunda clase de estadística }}
    \author{\spacedlowsmallcaps{Rodrigo Castillo}}
    \date{\today} 
    
    \maketitle
     

     % ====| Loguito |==== %
    \includegraphics[width=0.1\linewidth]{negro_cara.png}
    %=======================NOTES GOES HERE===================%
    \section{Enunciado}
        \includegraphics[width=0.8\linewidth]{enunciado_tarea.png}
    \section{punto a}
        \section{el vector está dado por :}
        \includegraphics[width=0.8\linewidth]{imagen_punto_a.png}
        \\ grafique los datos como un diagrama de dispersion y calcule s _{11} , s _{22} , s _{12}   
        \subsection{grafica:}
        \includegraphics[width=0.8\linewidth]{diagrama_dispersion_x1_x2}
        \subsection{s _{11} , s _{22} , s _{12}   }
        % ====| preguntarle a julian |==== %
    \section{punto b}
        usando \hat x
        
            

        







    %=======================NOTES ENDS HERE===================%
    
    % bib stuff
    \nocite{*}
    \addtocontents{toc}{\protect\vspace{\beforebibskip}}
    \addcontentsline{toc}{section}{\refname}    
    \bibliographystyle{plain}
    \bibliography{../Bibliography}
\end{document}
