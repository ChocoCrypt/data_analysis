% article example for classicthesis.sty
\documentclass[10pt,a4paper]{article} % KOMA-Script article scrartcl
\usepackage{import}
\usepackage{xifthen}
\usepackage{pdfpages}
\usepackage{transparent}
\newcommand{\incfig}[1]{%
    \def\svgwidth{\columnwidth}
    \import{./figures/}{#1.pdf_tex}
}
\usepackage{lipsum}     %lorem ipsum text
\usepackage{titlesec}   %Section settings
\usepackage{titling}    %Title settings
\usepackage[margin=10em]{geometry}  %Adjusting margins
\usepackage{setspace}
\usepackage{listings}
\usepackage{amsmath}    %Display equations options
\usepackage{amssymb}    %More symbols
\usepackage{xcolor}     %Color settings
\usepackage{pagecolor}
\usepackage{mdframed}
\usepackage[spanish]{babel}
\usepackage[utf8]{inputenc}
\usepackage{longtable}
\usepackage{multicol}
\usepackage{graphicx}
\graphicspath{ {./Images/} }
\setlength{\columnsep}{1cm}

% ====| color de la pagina y del fondo |==== %
\pagecolor{black}
\color{white}



\begin{document}
    %========================{TITLE}====================%
    \title{{  Taller 2 Analisis de datos  }}
    \author{{Rodrigo Castillo}}
    \date{\today}

    \maketitle


    %=======================NOTES GOES HERE===================%
    \section{Dada la matriz de datos ... }
        \begin{equation}
            X = \begin{pmatrix}
               9 & 1
               \\ 5 & 3
               \\ 1 & 2
            \end{pmatrix}
        \end{equation}

        \begin{enumerate}

            \item {grafique un diagrama de dispersion en $ p = 2  $ dimensiones
                . Localice la media de la muetra en su daigrama}

            \item {Dibuje la representacion $ n = 3  $ dimensional de los datos
                y trace los vectores de desviacion  $ y_1  \hat{x_1  }   $ y $ y_2 - \hat{x_2}   $ }

            \item {Dibuje los vectores de desviación en (b) que emanan del origen. Calcula las
            longitudes de estos vectores y el coseno del ángulo entre ellos. Relacione estas
            cantidades con $ S_n , R  $ }
            \item {\color{red} Calcular la varianza muestral generalizada $ \color{white}  S  $ }
        \end{enumerate}

        \section{demuestre que $ \color{blue} S \color{white}  = (s_{11} ,
        s_{22} , s_{33} , ... , s_{pp})\color{blue} R \color{white}   $ }
        \color{yellow} las cosas que están en \color{blue} azul \color{white} es porque tienen
        el simbolo ese de cardinal que no sé como guardar :( .

 \color{white}























    %=======================NOTES ENDS HERE===================%

    % bib stuff
    \nocite{*}
    \addtocontents{toc}{{}}
    \addcontentsline{toc}{section}{\refname}
    \bibliographystyle{plain}
    \bibliography{../Bibliography}
\end{document}
